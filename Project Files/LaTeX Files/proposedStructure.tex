\documentclass[12pt]{article} % use larger type; default would be 10pt


%%% PAGE DIMENSIONS
\usepackage[lmargin=.5in, rmargin=1.5in, marginpar=1.3in]{geometry} % to change the page dimensions (margins, etc.)
\geometry{letterpaper} % or letterpaper (US) or a5paper or....

\usepackage{graphicx} % support the \includegraphics command and options

% \usepackage[parfill]{parskip} % Activate to begin paragraphs with an empty line rather than an indent

%%% PACKAGES
\usepackage{setspace} % for changing line spacing
\usepackage{array} % for better arrays (eg matrices) in maths
\usepackage{paralist} % very flexible & customisable lists (eg. enumerate/itemize, etc.)
\usepackage{verbatim} % adds environment for commenting out blocks of text & for better verbatim
\usepackage{subfig} % make it possible to include more than one captioned figure/table in a single float
\usepackage{amsthm} % allows example and proof environment to be defined/used
\usepackage{amssymb}
\usepackage{todonotes}
%%% HEADERS & FOOTERS
\usepackage{fancyhdr} % This should be set AFTER setting up the page geometry
\pagestyle{fancy} % options: empty , plain , fancy
\renewcommand{\headrulewidth}{0pt} % customise the layout...
\lhead{}\chead{}\rhead{}
\lfoot{}\cfoot{\thepage}\rfoot{}



%%% ToC (table of contents) APPEARANCE
\usepackage[nottoc,notlof,notlot]{tocbibind} % Put the bibliography in the ToC
\usepackage[titles,subfigure]{tocloft} % Alter the style of the Table of Contents


%%% SECTION COUNTER SETTINGS
\newcommand{\mysection}[2]{\setcounter{section}{#1}\section*{Chapter #1: #2}} % changes a "1" to "Chapter 1," and so on, and allows you to number the chapters manually if necessary. 


%%% COMPLETE AND INCOMPLETE COMMANDS
\newcommand{\incomplete}{$\square$}
\newcommand{\complete}{$\boxtimes$}
\usepackage{ marvosym }
\newcommand{\graded}{\Smiley}

%%% DEFINE EXAMPLE AND THEOREM ENVIRONMENTS
\theoremstyle{definition}
\newtheorem{example}{Example}[section]
\newtheorem{theorem}{Theorem}

%%% END Article customizations

%%% The "real" document content comes below...

\title{PAIP GOes Haskell}
\author{Proposed Structure of the Wiki Page} % Fill in your name here
\date{} % Activate to display a given date or no date (if empty),
         % otherwise the current date is printed 

\begin{document}

\maketitle

%Here you (and I) will indicate when drafts are ready to be looked over and have been looked over. Change the \unmark before an item to \mark. 

\incomplete \hspace{7pt} Cover page: Name of the Project, Team Members, Date
\vspace{5pt}

\incomplete \hspace{7pt} Reference Book's name with a concise description of the book
\vspace{5pt}

\incomplete \hspace{7pt} Introduction 
\vspace{5pt}

\hspace{1cm}\incomplete \hspace{7pt} Goal/Purpose
\vspace{5pt}

\hspace{1cm}\incomplete \hspace{7pt} Acknowledgment
\vspace{5pt}

\incomplete \hspace{7pt} Table of Contents
\vspace{5pt}

\incomplete \hspace{7pt} Main Content*
\vspace{5pt}

\incomplete \hspace{7pt} Meet the Team

\vspace{1cm}
\textbf{Main Content*}
\vspace{3pt}

\incomplete \hspace{7pt} Page numbers (PAIP)
\vspace{5pt}

\incomplete \hspace{7pt} Problem Description with Lisp code (PAIP)
\vspace{5pt}

\incomplete \hspace{7pt} Link to Haskell/Go code in repository 
\vspace{5pt}

\hspace{1cm} \incomplete \hspace{7pt} Format \{ Approach \# → explanation: functions used \}
\vspace{5pt}

\incomplete \hspace{7pt} Embed links to Haskell’s official documentation, if necessary
\vspace{5pt}

\incomplete \hspace{7pt} Explanations, if necessary

\end{document}